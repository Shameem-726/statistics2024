\documentclass[10pt,letterpaper]{article}
\usepackage[margin=.75in]{geometry}
\usepackage{amsmath}
\usepackage{amssymb}
\usepackage{fancyhdr}
\usepackage{pgfplots}
\usepackage[shortlabels]{enumitem}
\usepackage{listings}
\usepackage[document]{ragged2e}
\usepackage{inconsolata}
\pgfplotsset{compat=1.16}
\usepackage{graphicx}
\usepackage{setspace}
\graphicspath{ {./images/} }
\onehalfspacing

\author{Gabriel Brown}
\title{Statistics For Data Science: HW 1}

\pagestyle{fancy}
\renewcommand{\headrulewidth}{0pt}
\renewcommand{\footrulewidth}{0pt}
\lstset{frame=tb,
  aboveskip=3mm,
  belowskip=3mm,
  showstringspaces=false,
  columns=flexible,
  basicstyle={\small\ttfamily},
  numbers=none,
  numberstyle=\tiny\color{gray},
  breaklines=true,
  breakatwhitespace=true,
  tabsize=3
}
\setlength{\headheight}{25pt}
\fancyhf{}
\rhead{
    Shameem Ahmed Khan | Orkun Akyol | Gabriel Brown\\
    Statistics | Winter 2024\\
    Homework 1
}
\rfoot{\thepage}

\begin{document}
\paragraph{1}
\textbf{A medical test for disease D has outcomes + (positive) and - (negative). We assume that \begin{itemize}
    \item the probability for an individual to have the disease is 0.01
    \item the probability of a positive test, given that the individual has the disease, is 0.9
    \item the probability of a negative test, given that the individual does not have the disease, is 0.9
\end{itemize}
Compute the probability that an individual has the disease, given that the
individual has tested positive. Comment on the quality of the test.}\\
P(has disease $|$ positive test) = P(positive test $|$ has disease) P(has disease) / P(positive test)\\
= $0.9 * 0.01 /$ P(positive test)\\
P(positive test) = P(positive test $|$ has disease) P(has disease) + P(positive test $|$ doesn't have disease) P(doesn't have disease)\\
= P(positive test $|$ has disease) P(has disease) + (1 - P(negative test $|$ doesn't have disease)) P(doesn't have disease)\\
= $0.9 * 0.01 + (1 - 0.9)*.99 = 0.108$\\
Therefore, the probability that a patient has the disease given a positive test = $0.9 * 0.01 / 0.108 = 0.0833$ or 8.33\%\\
Therefore, the test is not very useful in diagnosing a patient.

\paragraph{3}
\textbf{Let X be a continuous real-valued random variable with the probability density
function $f: \mathbb{R} \xrightarrow{} \mathbb{R}$,
$$f(x) = \begin{cases}
    0 & \text{ if } x < 0\\
    \frac{4xe^{-x^2}}{(1+e^{-x^2})^2} & \text{ if } x \geq 0
\end{cases}$$}\\
\begin{enumerate}[(a)]
    \item \textbf{Compute the cumulative distribution function $F(x) = \mathbb{P}(X \leq x)$ of $X$ for $x \in \mathbb{R}$}\\
    For all $x < 0$, $F(x) = 0$ trivially\\
    For all $x \geq 0, F(x) = \int_{0}^{x} f(u) du$\\
    $= \int_{0}^{x} \frac{4ue^{-u^2}}{(1+e^{-u^2})^2} du$\\
    $= \frac{-2}{e^{u^2}+1}|_{0}^{x} = \frac{e^{x^2}-1}{e^{x^2}+1} = \tanh{\frac{x^2}{2}}$\\
    Therefore,
    $F(x) = \begin{cases}
        0 & x < 0\\
        \tanh(\frac{x^2}{2}) & x \geq 0\\
    \end{cases}$

    \item \textbf{Solve the quantile function of X}\\
    Given $q \in (0,1), \exists x$ such that $F(x) = q$\\
    $\implies \tanh(\frac{x^2}{2}) = q, \forall q\in (0,1)$\\
    $\implies x = \sqrt{2\tanh^{-1}(q)}$\\
    $\implies F_X^{-1}(q)= \sqrt{2\tanh^{-1}(q)}$ for $q\in (0,1)$

    \item \textbf{Compute the probability $\mathbb{P}(0 < X < 1)$. Which value $a\in \mathbb{R}$ satisfies $\mathbb{P}(X \leq a) = 0.95$}\\
    $P(0 < X < 1) = F(1) - F(0) = \tanh(1/2) \approx 0.462$\\
    by the definition of the quantile function, $a = F_X^{-1}(.95) = \sqrt{2\tanh^{-1}(.95)} \approx 1.91404$
\end{enumerate}


\paragraph{4}
\begin{enumerate}[(a)]
    \item The sample space is the following set with four members: 
        \[
            \omega = \{(0,0), (1,0), (0,1), (1,1)\}
        \]
        By simple enumeration, it follows:\\ 
        \text{} \\
    $p_X(x) = \begin{cases}
        1/4 & \text{if } x = 0\\
        1/2 & \text{if } x = 1\\
        1/4 & \text{if } x = 2\\
    \end{cases}$
    \item Trivially, $F(x)=0$ for $x<0$ and $F(x)=1$ for $x \ge 2$. By cumulatively summing the probabilities in the probability mass function, we get: \\
    \text{} \\
    $F(x) = \begin{cases}
        0 & \text{if } x<0\\
        1/4 & \text{if } 0 \le x < 1\\
        3/4 & \text{if } 1 \le x < 2\\
        1 & \text{if } x \ge 2\\
    \end{cases}$
    \item $F^{-1}(q) = \begin{cases}
         0 & \text{if } 0 < q \le 1/4\\
         1 & \text{if } 1/4 < q \le 3/4\\
         2 & \text{if } 3/4 < q < 1\\
    \end{cases}$
\end{enumerate}
\end{document}