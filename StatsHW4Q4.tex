\documentclass{article}
\usepackage{graphicx} % Required for inserting images


\date{November 2024}

\begin{document}

\maketitle

\section{Question 1}
\textbf{(a)} Given that the sample mean \(\bar{m}_n\) is
\[
\bar{m}_n = \frac{1}{n} \sum_{k=1}^{n} X_k
\]
where \(X_1, X_2, \dots, X_n\) are i.i.d. random variables, each with mean \(E[X_k] = m\). 
For the linear expectation operator:
\[
E[\bar{m}_n] = E\left[\frac{1}{n} \sum_{k=1}^{n} X_k\right] = \frac{1}{n} \sum_{k=1}^{n} E[X_k] = \frac{1}{n} \cdot n \cdot m = m
\]
Thus, \(E[\bar{m}_n] = m\).

\vspace{1em} % Add space between paragraphs

\textbf{(b)} The variance of \(\bar{m}_n\) is:
\[
\text{Var}(\bar{m}_n) = \text{Var}\left(\frac{1}{n} \sum_{k=1}^{n} X_k\right)
\]
Using the property \(\text{Var}(aX) = a^2 \text{Var}(X)\), we get
\[
\text{Var}(\bar{m}_n) = \frac{1}{n^2} \text{Var}\left(\sum_{k=1}^{n} X_k\right)
\]
For independent random variables \(X_1, X_2, \dots, X_n\), the variance of the sum is the sum of their variances:
\[
\text{Var}(\bar{m}_n) = \frac{1}{n^2} \sum_{k=1}^{n} \text{Var}(X_k)
\]
Since all \(X_k\) are identically distributed with \(\text{Var}(X_k) = \sigma^2\), we have:
\[
\text{Var}(\bar{m}_n) = \frac{1}{n^2} \sum_{k=1}^{n} \sigma^2 = \frac{1}{n^2} \cdot n \cdot \sigma^2 = \frac{\sigma^2}{n}
\]
Thus, \(\text{Var}(\bar{m}_n) = \frac{\sigma^2}{n}\).

\vspace{1em} % Add space between paragraphs

\textbf{(c)} The general form of the normal distribution is:
\[
f(x) = \frac{1}{\sqrt{2 \pi \sigma^2}} e^{-\frac{1}{2 \sigma^2} (x - \mu)^2}
\]
Simplifying the provided probability density function to the general form for the normal distribution:
\[
f(x) = \frac{1}{\sqrt{4 \pi}} e^{-\frac{1}{4} (x - 1)^2} = \frac{1}{\sqrt{2 \pi 2}} e^{-\frac{1}{2 \cdot 2} (x - 1)^2}
\]
So, the provided probability density function is a normal distribution with mean \(\mu = 1\) and variance \(\sigma^2 = 2\).

Given that
\[
\text{Var}(\bar{m}_n) \leq 10^{-2}
\]
\[
\frac{\sigma^2}{n} \leq 10^{-2}
\]
we have ${\sigma^2 =2}$
\[
\frac{2}{n} \leq 10^{-2}
\]
which implies
\[
n \geq 2 \times 10^2 = 200
\]
Thus, the sample size \(n\) needs to be 200 or greater.

\end{document}

\section{Question 4}
\section{(a)}
To show that \(f_k\) is a probability density function, these conditions have to hold:
\[f_k(x) \geq 0, \forall x \in (0,\infty) \tag{1}\]
\[
\int_{-\infty}^{\infty} f_k(x)dx =1 \tag{2}
\]
The (1) is proved because for \(x<1\), \(f_k(x)=0\) and for \(x \geq 1\), the sign of the function only depends on \(k\), which is non-negative because it is a natural number.
The (2) is proved because: \[
\int_{-\infty}^{1} 0 dx + \int_{1}^{\infty}\frac{k}{x^{k+1}}dx=k\int_{1}^{ \infty}x^{-k-1}dx=1, \forall k>0 \tag{3}
\]
\section{(b)}
To prove the existence of the moments, we need to proof that they are finite. Therefore:
\[
\mathbb{E}[X^ℓ] = \int_{1}^{\infty}\frac{kx^ℓ}{x^{k+1}}dx=k\int_{1}^{\infty}x^{ℓ-k-1}dx \tag{4}
\]
To be finite, the integrand have to converge, which happens when \(ℓ-k-1<-1\), or \(ℓ<k\). This shows that the moments \(\mathbb{E}[X^ℓ]\) exist for \(0 \leq ℓ\leq k-1\) and not for \(ℓ\geq k\).
\section{(c)}
Each of the random variables \(X_i\) is integrable if its expected value is finite. Noting (4):\[
\mathbb{E}[X]=k\int_{1}^{\infty}x^{-k}dx \tag{5}
\]
we can say that \(\mathbb{E}[X]\) is finite when \(-k<-1\). Considering that the random variables \(X_i\) are i.i.d. by construction, the weak Law of Large Numbers suggests that the sample average \(\overline{X}_n\) converges in probability to \(\mathbb{E[X]}\) for \(k>1\). For \(k=1\), \(\mathbb{E}[X]\) does not exist.
\end{document}
