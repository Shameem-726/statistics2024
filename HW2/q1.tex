\documentclass[10pt,letterpaper]{article}
\usepackage[margin=.75in]{geometry}
\usepackage{amsmath}
\usepackage{amssymb}
\usepackage{fancyhdr}
\usepackage{pgfplots}
\usepackage[shortlabels]{enumitem}
\usepackage{listings}
\usepackage[document]{ragged2e}
\usepackage{setspace}
\onehalfspacing

\author{Orkun Akyol}
\title{Statistics For Data Science: HW 2}

\pagestyle{fancy}
\renewcommand{\headrulewidth}{0pt}
\renewcommand{\footrulewidth}{0pt}
\lstset{frame=tb,
  aboveskip=3mm,
  belowskip=3mm,
  showstringspaces=false,
  columns=flexible,
  basicstyle={\small\ttfamily},
  numbers=none,
  numberstyle=\tiny\color{gray},
  breaklines=true,
  breakatwhitespace=true,
  tabsize=3
}
\setlength{\headheight}{25pt}
\fancyhf{}
\rhead{
    Statistics | Winter 2024\\
    Homework 2
}
\rfoot{\thepage}

\begin{document}

\paragraph{Q1}
    To use the inverse transform sampling, first we need to calculate the cumulative distribution function of X. 
    \[
        F(x) = \int_{0}^{x} (4te^{-t^2})/(e^{-t^2}+1)^2 \,dt \
    \]    
    To calculate the integral, let us make the substitution \(u = e^{-t^2}+1 \). It follows that \(du = -2te^{-t^2}+1 \). Then,

    \[
        F(x) = -2\int_{0}^{x} 1/u^2 \,du \
    \]   
    \[
        = 2/u \bigg\rvert_0^x
    \] 
    \[
        = 2/(e^{-t^2}+1) \bigg\rvert_0^x
    \]  
    \[
        = 2/(e^{-x^2}+1)-1
    \]          
    \[
        = (e^{t^2}-1)/(e^{t^2}+1)
    \]         
        More precisely: \\
        \text{} \\
    $F(x) = \begin{cases}
        0 & \text{if } x < 0\\
        (e^{t^2}-1)/(e^{t^2}+1) & \text{if } x \ge 0\\
    \end{cases}$ \\
        \text{} \\

    Now, let us calculate the inverse of the cdf: 
    
     \[
        y = (e^{t^2}-1)/(e^{t^2}+1)
    \]    
    \[
        y * (e^{t^2}+1) = (e^{t^2}-1) 
    \]  
    \[
        y * e^{t^2} - (e^{t^2}) = -1-y 
    \]
    \[
        e^{t^2} * (y-1) = -1-y
    \]
    \[
        t^2 = \ln((-1-y)/(y-1))
    \]
        \[
        t = \sqrt{\ln((-1-y)/(y-1))}
    \]
    Since the pdf is $0$ for $x<0$, we are interested in the positive root. For sampling from the CDF, the outputs and the plots, please see HW2Q1.ipynb. 

\end{document}
